\NeedsTeXFormat{LaTeX2e}
\documentclass[a4paper,12pt,
headsepline,           % Linie zw. Kopfzeile und Text
oneside,               % einseitig
pointlessnumbers,      % keine Punkte nach den letzten Ziffern in Überschriften
bibtotoc,              % LV im IV
%DIV=15,               % Satzspiegel auf 15er Raster, schmalere Ränder   
BCOR15mm               % Bindekorrektur
%,draft
]{scrbook}
\KOMAoptions{DIV=last} % Neuberechnung Satzspiegel nach Laden von Paket helvet

\pagestyle{headings}
\usepackage{blindtext}

% für Texte in deutscher Sprache
\usepackage[english]{babel}
\usepackage[utf8]{inputenc}
\usepackage[T1]{fontenc}

\usepackage{xcolor}

\definecolor{textblue}{rgb}{.2,.2,.7}
\definecolor{textred}{rgb}{0.54,0,0}
\definecolor{textgreen}{rgb}{0,0.43,0}
\usepackage{listings}
\lstloadlanguages{C,C++}
\lstset{language=[11]C++}
\lstset{morekeywords={concept}}
\lstset{breaklines=true,numbersep=10pt,stepnumber=1,captionpos=t}
\lstset{basewidth={0.60em,0.48em},columns=[l]fullflexible,flexiblecolumns=true}
\lstset{frame=tb,numbers=left,numberstyle=\tiny,escapechar=`,extendedchars=true}
\lstset{basicstyle={},%
	keywordstyle=\bfseries,
	identifierstyle=\itshape,
	commentstyle=\color{gray}\itshape,
	stringstyle=}

% Helvetica als Standard-Dokumentschrift
\usepackage[scaled]{helvet}
\renewcommand{\familydefault}{\sfdefault} 

\usepackage{graphicx}

% Literaturverzeichnis mit BibLKaTeX
\usepackage[babel,german=quotes]{csquotes}
\usepackage[backend=bibtex8]{biblatex}
\bibliography{bibliography}

% Für Tabellen mit fester Gesamtbreite und variabler Spaltenbreite
\usepackage{tabularx} 

% Besondere Schriftauszeichnungen
\usepackage{url}              % \url{http://...} in Schreibmaschinenschrift
\usepackage{color}            % zum Setzen farbigen Textes

\usepackage{amssymb, amsmath} % Pakete für Mathe-Umgebungen und -Symbole

\usepackage{setspace}         % Paket für div. Abstände, z.B. ZA
%\onehalfspacing              % nur dann, wenn gefordert; ist sehr groß!!
\setlength{\parindent}{0pt}   % kein linker Einzug der ersten Absatzzeile
\setlength{\parskip}{1.4ex plus 0.35ex minus 0.3ex} % Absatzabstand, leicht variabel

% Tiefe, bis zu der Überschriften in das Inhaltsverzeichnis kommen
\setcounter{tocdepth}{3}      % ist Standard


% hier Namen etc. einsetzen
\newcommand{\fullname}{Adrian Lauber}
\newcommand{\email}{adrian.lauber@uni-ulm.de}
\newcommand{\titel}{C++: Solving Markov Decision Processes}
\newcommand{\jahr}{2020}
\newcommand{\matnr}{1042606}
\newcommand{\gutachterA}{Prof.\,Dr.\,Andreas Borchert}
\newcommand{\gutachterB}{}
\newcommand{\betreuer}{}
% hier die Fakultät auswählen
%\newcommand{\fakultaet}{---  Im Quellcode anpassen nicht vergessen! ---}
%\newcommand{\fakultaet}{Ingenieurwissenschaften, Informatik und\\Psychologie}
\newcommand{\fakultaet}{Mathematik und\\Wirtschafts-\\wissenschaften}
%\newcommand{\fakultaet}{Medizin}
%\newcommand{\fakultaet}{Naturwissenschaften}

% hier das Institut einsetzen
\newcommand{\institut}{Institut für numerische Mathematik}

% Informationen, die LaTeX in die PDF-Datei schreibt
\pdfinfo{
  /Author (\fullname)
  /Title (\titel)
  /Producer     (pdfeTex 3.14159-1.30.6-2.2)
  /Keywords ()
}

\usepackage{hyperref}
\hypersetup{
pdftitle=\titel,
pdfauthor=\fullname,
pdfsubject={Diplomarbeit},
pdfproducer={pdfeTex 3.14159-1.30.6-2.2},
colorlinks=false,
pdfborder=0 0 0	% keine Box um die Links!
}

% Trennungsregeln
\hyphenation{Sil-ben-trenn-ung}

\begin{document}
\frontmatter

% Titelseite
\thispagestyle{empty}
\begin{addmargin*}[4mm]{-10mm}

\includegraphics[height=1.8cm]{images/unilogo_bild}
\hfill
\includegraphics[height=1.8cm]{images/unilogo_wort}\\[1em]

{\footnotesize
%{\bfseries Universität Ulm} \textbar ~89069 Ulm \textbar ~Germany
\hspace*{115mm}\parbox[t]{35mm}{\bfseries Fakultät für\\
\fakultaet\\
% TODO hier Institut anpassen
\mdseries \institut}\\[2cm]

\parbox{140mm}{\bfseries \LARGE \titel}\\[2.5em]
{\footnotesize Prüfungsleistung für die Universität Ulm}\\[3em]

{\footnotesize \bfseries Vorgelegt von:}\\
{\footnotesize \fullname\\ \email}\\ \matnr\\[2em]
{\footnotesize \bfseries Gutachter:}\\                     
{\footnotesize \gutachterA\\ \gutachterB}\\[2em]
%{\footnotesize \bfseries Betreuer:}\\ 
%{\footnotesize \betreuer}\\\\
{\footnotesize \jahr}
}
\end{addmargin*}


% Impressum
\clearpage
\thispagestyle{empty}
{ \small
  \flushleft
  Fassung \today \\\vfill
  \copyright~\jahr~\fullname\\[0.5em]
% Wenn Sie Ihre Arbeit unter einer freien Lizenz bereitstellen möchten, können Sie die nächste Zeile in Ihren Code aufnehmen. Bitte beachten Sie, dass Sie hierfür an allen Inhalten, inklusive enthaltener Abbildungen, die notwendigen Rechte benötigen! Beim Veröffentlichungsexemplar Ihrer Dissertation achten Sie bitte darauf, dass der Lizenztext nicht den Angaben in den Metadaten der genutzten Publikationsplattform widerspricht. Nähere Information zu den Creative Commons Lizenzen erhalten Sie hier: https://creativecommons.org/licenses/
%This work is licensed under the Creative Commons Attribution 4.0 International (CC BY 4.0) License. To view a copy of this license, visit \href{https://creativecommons.org/licenses/by/4.0/}{https://creativecommons.org/licenses/by/4.0/} or send a letter to Creative Commons, 543 Howard Street, 5th Floor, San Francisco, California, 94105, USA. \\
  Satz: PDF-\LaTeXe
}

% ab hier Zeilenabstand etwas größer 
\setstretch{1.2}

\tableofcontents

\mainmatter
\chapter{Introduction}


The goal of this project is to implement a library to solve \emph{Markov decision processes (MDPs)}. The focus is on extensibility, facilitating modern C++ features like smart pointers as well as following well-established C++ guidelines and best-practices. 

The first chapter is a brief introduction into the problem class for which the library functionality is developed.  

In the second chapter the focus is on the actual implementation. The class hierarchy is described as aspects of the build and test environment. 

\chapter{Markov Decision Process}

Markov Decision Processes describe a time-discrete and stochastic process that can be used to model different decision-making problems. MDPs are used in all kinds of domains and multiple algorithms have been proposed to solve a particular MDP. In the field of Reinforcement Learning MDPs are a tool \cite{Sutton2018} to describe the problem of learning. \emph{Agent-Environment interface} where the agent interacts with the environment  is to find an optimal strategy with respect to a defined 

\section{Definition}

A MDP is defined by a state-transition-function $P_a(s,s')$ that gives the probability of transferring to a consecutive state $s'$ when in state $s$ and applying action $a$. For a given state $s$ and action $a$ the probability is independent from former states or actions. This property is referred to as the \emph{Markov property}. The goal of optimization is a scalar value referred to as the \emph{cost} or \emph{reward} which is either minimized (in case a problem defined by costs) or maximized (for rewards). This optimization metric is a function of the current state $s$ and consecutive state $s'$ given an action: $R_a(s,s')$

\section{Algorithms}

For finite state and action spaces 
\chapter{Implementation}

\section{Design goals}
\label{goals}

Considering potential use cases is an important step when creating the interface and architecture of a library. The purpose of the developed library is to solve a defined decision making problem and in order to do that following input is needed:

\begin{itemize}
	\item Definition of the Markov decision process
	\item Configuration and parameters for solving
\end{itemize}

For the definition of the MDP there are two main use cases. It has either been formally described in a file or it lives within an application. For the first use case it has been decided that the library shall provide functionality to parse a MDP from a file and that additional parsers for other formats can be easily added. For the other use case the library shall expose the model class in the interface so that an external application can create and setup a MDP and then pass it to the library for solving. For this use case a wrapper class would have been an alternative to directly exposing the model class. This alternative has drawbacks due to the different representations possible and also the sizes of the data structures that need to defined. For example providing a wrapper method to set the probability value of a single state transition for a given action would require infeasible many calls for a MDP with a large state and action space. 

A structure containing all necessary parameters shall be exposed in the interface so that the application can setup the structure to its needs and then pass it for solving. 

After solving the typical use case is to analyze the resulting strategy and/or deploy it to the environment it has been created for. This requirement shall be fulfilled by transferring the ownership to an object containing the policy. Owning this object shall enable the application to draw actions for a given state. This requirement is especially crucial for stochastic policies, where the policy object also depends on random number generators. Providing the application with the ownership of a policy enables easy deployment of the optimized strategy. An alternative would have been the transfer of a data structure defining the resulting stochastic or deterministic policy. In case of a stochastic policy this can be a heavily complex data structure (for example when using neural networks as a policy) and has consequently not been considered. 
Since many different solving methods exist, new functionality shall be easy to integrate.
The library shall guarantee \emph{basic exception safety}. 

\section{Interface}

Considering the requirements from \autoref{goals} the header \emph{interface.hpp} is provided to the application:

\begin{lstlisting}
struct Params{
	std::string mdp_filepath_;
	std::string module_parser_;
	std::string module_eval_;
	std::string module_policy_;
	std::string module_solver_;
	std::size_t solver_iteration_cnt_;
};
/*
Solve mdp that is defined in a file on the filesystem
*/
std::unique_ptr<Policy> solve_filebased_mdp(const Params& params);
/*
Solve mdp that has been defined by the application
*/
std::unique_ptr<Policy> solve_external_mdp(Model& model,const Params& params);

\end{lstlisting}

The functionality provided in the interface is implemented in \emph{solve.cpp}. 

\section{Model}

The model class is the core element of library because it holds information about the MDP to be solved. It is initialized using the default constructor and a Parser-Module (\autoref{chaptermodule}) is parsing the MDP definition from the input file to the model class. Since memory for the model class members has to be allocated dynamically, the parsing model class object offers a public method to allocate the required memory for the object. Once the Parser-Module has determined the size of the state and action space, \emph{setArrays()} can be called to initialize the array data structures of the state transition matrix and the reward matrix. A method to validate the model data is offered by the model class and is called by the Parser-Module to abort parsing if an inconsistency has been detected. 

The state transition matrix and reward matrix are represented using the \emph{Array3d template class} (\autoref{array}).
Both matrices are used to look up transition probabilities and rewards.

\begin{equation}
P_a(s,s') = state\_transition\_matrix[a,s,s']
\end{equation}

\begin{equation}
R_a(s,s') = reward\_matrix[a,s,s']
\end{equation}



\section{Modules}
\label{chaptermodule}

In order to achieve a high level of extensibility this  base class acts a blue print for classes to read or write the model. Using this common interface enables easy integration which is described in detail in \autoref{integration}. The module class consists of a protected constructor to set its most important member: a reference to the model. 

Four interfaces inherit from the module class and describe different types of modules:

\begin{itemize}
	\item Parser (parser.hpp)
	\item Policy (policy.hpp)
	\item Evaluation (evaluation.hpp)
	\item Solver (solver.hpp)
\end{itemize}

The abstract \emph{Parser} class acts as an interface for concrete Parser implementations. There exist several file formats to describe MDPs. For the sake of a running example, a rudimentary implementation to parse \emph{POMDP} file format \autocite{Cassandra} is provided. This class only parses a minimal subset of the POMDP grammar and lacks validation. 

The abstract \emph{Policy} class as a interface for different variations of \emph{policy improvement}. The \emph{policy mapping} member is a multdimensional matrix represented as an \emph{Array3d Object} (\autoref{array}) in order to both support deterministic policies and stochastic policies. A method is provided to select an action given a certain state. This is needed by Evaluation-implementations. The policy class is also used as a return type for the external interface of the library. This way an external application can use the reference to directly integrate the optimal strategy for the environment. 

The abstract \emph{Evaluation} class is an interface for different implementatons of policy evaluation. A simple form of policy evaluation is provided in the library: \emph{iterative policy evaluation}. This method applies \autoref{valuefunction} to update the value of each state. 

The \emph{Solver} interface and base class provides a general solve method. Solving algorithms combine policy evaluation and policy improvement in different ways to find the optimal policy. \emph{Policy iteration} is implemented as an example.

Classes implementing the required interface of these interfaces are located in the according subdirectories. 

\section{Module factory}
\label{integration}

In order to minimize the amount of effort to integrate for example an additional Parser or Solver, a factory (\emph{factory.hpp}) is implemented. This enables integration without adding any references in existing code. To ensure scalability it has been decided to use a template class as a factory and instantiate a factory for each module category (Parser, Policy, Evaluation, Solver). Based on the factory template, customized factories can be implemented if needed. The factory template class is designed as a singleton and follows the \emph{construct on first use idiom}. A given name defined in the constructor of a new module is used as an identifer within the factory to decide which object to create. Using multiple factories reduces the risk of name conflicts between modules.  Alternatively a global factory could have been used with the category name encorporated in the given name. A global factory would have offered lower flexibility and was disregarded. 

The constructor (\emph{constructor.hpp})is a template class and used as a type in the factory as well as a base class for concrete constructors. The concrete constructor class has to override the virtual create(..) method and register to its factory.
In order to minimize necessary code changes when including a new type, it has been decided to implement self registering types by instantiating a static constructor object. This has drawbacks when it comes to building since the static objects and new functionality remains unreferenced and dropping it has to be avoided as described in \autoref{build}. 
It has been decided that the advantage of not having to touch any existing code overweighs. 

\autoref{SolverFactory} depicts a factory of type Solver and a new Solver implementation ("NewSolver"). 

\begin{figure}[ht]
	\centering
	\includegraphics[width=.6\textwidth]{images/SolverFactory.png}
	\caption{\label{fig:bild2}Solver factory}
	\label{SolverFactory}
\end{figure}

For each new module two classes have to be added:

\begin{itemize}
	\item Class with new functionality, inheriting from module class or a module category
	\item Constructor class, inheriting from Constructor template class
\end{itemize}

After the constructor is succesfully registered, the factory can call its create(..) method which returns a pointer to the new object. In order to avoid memory leaks, the factory creates a std::unique\_ptr object for the returned pointer. The factory then returns this std::unique\_ptr to the caller which will also transfer exclusive ownership. By using a unique\_ptr the caller does not have to handle deletion of the created object. The drawback of this solution is that if the caller happens to need a std::shared\_ptr for distribution the construction of it comes at the cost of a dynamic memory allocation. 
Since the unique\_ptr is more lightweight and no use case for a shared\_ptr could be identified, it has been favored. 

Both constructor and factory enforce that new classes inherit from the module class when using the module factory. This is achieved with a new C++20 feature named \emph{concepts}. Concepts ease the implementation of compile-time validation of template arguments and provide meaningful compile errors to the user. 
Following concept enforces the template argument to inherit from the module class:

\begin{lstlisting}
template <typename T>
concept ModularType  = std::is_base_of<Module,T>::value;
\end{lstlisting}

The concept of a \emph{ModularType} is used for the factory and constructor template classes. 

\section{Array3d}
\label{array}
To process the state transition matrix and reward structure of a MDP, a multidimensional array implementation is needed. One option is to use \emph{std::vector} holding another std::vector for the two-dimensional case. The std::vector class manages its own resources following RAII (Resource Acquisition Is Initialization)making this a simple and robust solution but performance-wise there is a drawback. The memory for this nested structure will be fragmented which will slow down access. 
In order to make use of std::vector without causing fragmented memory the \emph{Array3d} template class (\emph{array3d.hpp}) acts as a wrapper to a one-dimensional std::vector. The () operator accepts three indices and maps those to the 1d std:.vector member. The wrapper performs a boundary check on the 3d indices. If successful, the access to the 1d vector can be performed without the need for a boundary check. This enhances the performance.  

The Array3d template class does not constrain the template argument in any way. Since internally a std::vector object with the same type has to be instantiated, the restrictions that the vector class provides are sufficient. 

\section{Example}

A sample mdp is provided in the \emph{data} directory: "4x3.95.POMDP" \autocite{Cassandra}. The mdp describes a 4x3 gridworld (\autoref{maze}) with a blocked cell (\#), a cell with high positive reward(+1) and a cell with high negative reward(-1). The cell field is not in the state space and the states of the cells with high rewards cause the transition to a random state (restarting). The action space consists of four move actions: north(N), south(S), east(E), west(W). Moving into the wall or into the blocked cell will return the same state. With a probability of 20\% an move into a direction which is perpendicular to the intended direction is happening. The mdp is used with full observability. 

The example application (directory: \emph{examples}) performs 20 steps of policy iteration on the defined mdp and then prints the expected optimal policy (\autoref{mazesolution}).

\begin{table}[]
	\centering
	\begin{tabular}{|c|c|c|c|}
		\hline
		-0.04 & -0.04                      & -0.04 & \cellcolor[HTML]{9AFF99}+1 \\ \hline
		-0.04 & \cellcolor[HTML]{C0C0C0}\# & -0.04 & \cellcolor[HTML]{FFCCC9}-1 \\ \hline
		-0.04 & -0.04                      & -0.04 & -0.04                      \\ \hline
	\end{tabular}
	\caption{4x3.95.POMDP by Stuart Russel}
	\label{maze}
\end{table}


% Please add the following required packages to your document preamble:
% \usepackage[table,xcdraw]{xcolor}
% If you use beamer only pass "xcolor=table" option, i.e. \documentclass[xcolor=table]{beamer}
\begin{table}[]
	\centering
	\begin{tabular}{|c|c|c|c|}
		\hline
		E & E                          & E & \cellcolor[HTML]{9AFF99}N \\ \hline
		N & \cellcolor[HTML]{C0C0C0}\# & N & \cellcolor[HTML]{FFCCC9}N \\ \hline
		N & E                          & N & W                         \\ \hline
	\end{tabular}
	\caption{Solution of the example mdp}
	\label{mazesolution}
\end{table}

\section{Build}
\label{build}

A makefile is used for compiling the library and linking it to the example (binary: example) as well as the unit tests (binary: tests). The compiler is set to \emph{-std=c++20} since \emph{Concepts} are a \emph{C++20} feature. Binaries have to be executed in the main directory (./bin/example) since relative paths are used for solving a mdp from file. 

Due to the modular concept, the concrete implementations are not referenced anywhere in the rest of the codebase. Since registration of all modules is done via instantiating unreferenced, static variables, the linker has to be told to not drop any unreferenced symbols. This is achieved using the \emph{-whole-archive} flag. The linker on macOS requires a different flag. This is considered by setting an OS variable when executing make: \emph{make OS=osx}. Using this option causes the binary to be bloated. For lightweight binaries the modules have to registered in the main code base in order to allow the linker to drop unreferenced symbols. 


\section{Testing}

\emph{Catch2}\autocite{Catch} has been chosen as a test framework. It is a header only library which eases integration. Test cases for each class are located in the same directory with a \emph{\_t.cc} suffix. The build process generates an executable containing all tests in the \emph{bin} directory. Overall 62 assertions in 9 different test cases are evaluated. 
Testing the \emph{concepts and constraints} is split up from the overall testing since the expected outcome is a compile-timer error. Manual compilation and review of following file is required to validate the introduced concept: concepts\_t.cxx.

\chapter{Extensibility}

\section{Modules}

The concept of modules enables to easily integrate additional functionality. By forcing new classes to inherit from module, information that is common over all modules can be easily added, for example an internal version information. The integrated module categories (Solver, Policy,...) provide interfaces in order to avoid any changes in the existing code base. 

\section{Partial Observability}

One variant of MDPs are partial observable Markov decision processes, abbreviated by \emph{POMDP}. In addition to a MDP, the state cannot be directly observed. Instead \emph{observations} are accessible which can be assigned to a certain state with some probability. By inheriting from the Model class a POMDP class can be easily integrated. 

\section{Reinforcement Learning}

Reinforcement learning problems can also be defined as a MDP but with (at least partially) unknown state transition behavior. A common approach is to first learn the state transition behavior by drawing samples from an environment and then solving the defined MDP. Reinforcement learning algorithms can be added to the Solver module category. A new module category with different environment implementations can manage sample data that is generated by rolling out a policy on the defined environment. This extension is compatible to the current architecture.  
% hier weitere Kapitel einbinden

\appendix
% hier Anhänge einbinden
%\input{chapters/sources}

\backmatter

\printbibliography

\clearpage
\thispagestyle{empty}

%Name: \fullname \hfill Matrikelnummer: \matnr \vspace{2cm}
%
%\minisec{Erklärung}
%
%Ich erkläre, dass ich die Arbeit selbständig verfasst und keine anderen als die angegebenen Quellen und Hilfsmittel verwendet habe.\vspace{2cm}
%
%Ulm, den \dotfill
%
%\hspace{10cm} {\footnotesize \fullname}
\end{document}

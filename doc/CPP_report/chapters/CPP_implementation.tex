\chapter{Implementation}

\section{Model}

The model class is the core element of library because it holds information about the MDP to be solved. It is initialized using the default constructor and a Parser-Module (\autoref{chaptermodule}) is parsing the MDP definition from the input file to the model class. Since memory for the model class members has to be allocated dynamically, the parsing model class object offers a public method to allocate the required memory for the object. Once the Parser-Module has determined the size of the state and action space, \emph{SetArrays()} can be called to initialize the array data structures of the state transition matrix and the reward matrix. A method to validate the model data is offered by the model class and is called by the Parser-Module to abort parsing if an inconsistency has been detected. 

//MODEL UML

\section{Modules}
\label{chaptermodule}

In order to achieve a high level of extensibility this a base class acts a blue print for classes to read or write the model. Using this common interface enables easy integration which is described in detail in \autoref{integration}. The module class consists of a protected constructor to set its only member: a reference to the model. 

The abstract \emph{Parser} class acts as an interface for concrete Parser implementations. There exist several file formats to describe MDPs. For the sake of a running example, a rudimentary implementation to parse \emph{POMDP} file format \autocite{Cassandra} is provided. 

The abstract \emph{Policy} class as a interface and base class for different variations of \emph{policy improvement}. The \emph{policy mapping } member is an Array3D object in order to both support deterministic policies and stochastic policies. A method is provided to select an action given a certain state. This is needed by Evaluation-implementations. The policy class is also used as a return type for the external interface of the library. This way an external application can use the reference to directly integrate the optimal strategy for the environment. 
The simplest form of policy improvement where always the most promising action is selected has been implemented. 

The abstract \emph{Evaluation} class is also both interface and base class for different ways to evaluate a policy. A simple form of policy evaluation is provided in the library: \emph{iterative policy evaluation}. This method applies \autoref{valuefunction} to update the value of each state. 

The \emph{Solver} interface and base class provides a general solve method. Solving algorithms combine policy evaluation and policy improvement to find the optimal policy. \emph{policy iteration/monte carlo planning} is implemented as an example.

\section{Module factory}
\label{integration}

In order to minimize the amount of effort to integrate for example an additional Parser or Solver, a factory is implemented. This enables integration without adding any references in existing code. To ensure scalability for 

For each new module two classes are necessary.

\begin{itemize}
	\item Class with new functionality, inheriting from Module Class
	\item Constructor class, inheriting from Constructor template
\end{itemize}

\subsection{Constructor}
Each new module 


\subsection{Factory}


\section{Array3D}

To process the state transition matrix and reward structure of a MDP, a multidimensional array implementation is needed. One option is to use \emph{std::vector} holding another std::vector for the two-dimensional case. The std::vector class manages its own resources following RAII (Resource Acquisition Is Initialization)making this a simple and robust solution but performance-wise there is a drawback. The memory for this nested structure will be fragmented which will slow down access. 
In order to make use of std::vector without causing fragmented memory the \emph{Array3d} template class acts as a wrapper to a one-dimensional std::vector. The () operator accepts three indices and maps those to the 1d std:.vector member. The wrapper performs a boundary check on the 3d indices. If successful, the access to the 1d vector can be performed without the need for a boundary check. This further enhances the performance.  

\section{Build}

A makefile is used for compiling the library and linking it to the example (binary: example) as well as the unit tests (binary: tests). The compiler is set to \emph{-std=c++20} since \emph{Concepts} are a \emph{C++20} feature. 

Due to the modular concept, the concrete implementations are not referenced anywhere in the rest of the codebase. Since registration of all modules is done via instantiating unreferenced, static variables, the linker has to be told to not drop any unreferenced symbols. This is achieved using the \emph{-whole-archive} flag. The linker on macOS requires a different flag. This is considered by setting an OS variable when executing make: \emph{make OS=osx}.



\section{Testing}

\emph{Catch2}\autocite{Catch} has been chosen as a test framework. It is a header only library which eases integration. Test cases for each class are located in the same directory with a \emph{\_t.cc} suffix. The build process generates an executable containing all tests in the \emph{bin} directory. Overall 48 assertions in 8 different test cases are evaluated. 
Testing the \emph{concepts and constraints} is split up from the overall testing since the expected outcome is a compile-timer error. Manual compilation and review of following file is required to validate the introduced concept: concepts\_t.cxx.
